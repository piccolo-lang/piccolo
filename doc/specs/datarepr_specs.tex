\documentclass[a4paper,11pt]{article}

\usepackage[utf8]{inputenc}
\usepackage[T1]{fontenc}
\usepackage{color}
\usepackage{graphicx}
\usepackage[pdfborder = {0 0 0}]{hyperref}
\usepackage{multicol}
\usepackage{multirow}

\definecolor{compilecolor}{gray}{0.45}
\definecolor{synchrocolor}{rgb}{0.1, 0.2, 0.8}
\title{PiThread Data Representation Specifications}
\date{\today}
\author{Aurélien Deharbe and Fr\'ed\'eric Peschanski\\ UPMC -- LIP6 -- APR}

\newenvironment{program}{
  \begin{sffamily}
  \begin{scriptsize}
  \begin{tabbing}}
 {\end{tabbing}
  \end{scriptsize}
  \end{sffamily}}

\newcommand{\code}[1]{\textsf{#1}}
\newcommand{\kw}[1]{\textsf{\textbf{#1}}}
\newcommand{\pindent}{\hspace{2em}\=}
\newcommand{\compiletime}[1]{\textcolor{compilecolor}{#1}}
\newcommand{\synchro}[1]{\textcolor{synchrocolor}{#1}}

\newcommand{\algotitle}[1]{\noindent\\ \noindent#1\par\nobreak\vspace{3pt}\hrule\vspace{6pt}}
\newcommand{\algosection}[1]{
  \phantomsection
  \addcontentsline{toc}{subsection}{#1}
  \algotitle{#1}
}
\newcommand{\myref}[1]{
  \hyperref[#1]{#1}
}

\begin{document}

\renewcommand{\contentsname}{Table of contents}
\maketitle
$ $\newline
$ $\newline
\begin{center}
\includegraphics[scale=0.45]{pithreads.png}
\end{center}
\newpage
\tableofcontents
\newpage

%%%%%%%%%%%%%%% Section COMMENTS %%%%%%%%%%%%%%%%%%
\section{Overview}

This document describes the low-level representation of values manipulated
by the pithreads backend compiler and runtime system.

There are 5 distinct categories of values manipulated in the pithreads:
\begin{itemize}
\item immediate values
\item immutable heap-allocated values: essentially tuples
\item managed values, heap-allocated and GC-friendly: e.g. strings and channels 
\item user-defined types: immediate or managed
\end{itemize}


%%%%%%%%%%%%%%% Section Tagged representation %%%%%%%%%%%
\section{Tagged representation}

The common representation of  values is as follows:

\begin{tabular}{|c|c|c|}
\hline
\code{tag} & \code{control} & \code{data} \\
\hline
\multicolumn{2}{|c|}{1 word} & $\geq 0$ word(s) \\
\hline
\end{tabular}

with:

\begin{itemize}
\item the tag part uniquely identifies the representation type of the value. The tag is mandatory.
\item the control part provide control-level informations about the value. The control is optional.
\item the data part provide data-level informations about the value. The data part is optional.
\end{itemize}

The exact number of bits occupied by a value is not strictly enforced, 
however, the tag part requires at least 8 bits and share with the control part at most a machine word (32bits or 64bits depending on the architecture).
The data part is of an arbitrary number of machine words, but essential 1 and 2 words.

There are rooms for 256 distinct tags, with
\begin{itemize}
\item tag 0 is reserved
\item tag 1 is novalue
\item tag 2 is boolean
\item tag 3 is integer
\item tag 4 is float
\item tag 64 is tuple
\item tag 128 is string
\item tag 253 is channel
\item tag 254 is user-defined / immediate
\item tag 255 is user-defined / managed
\end{itemize}

Tags 0-253 are for kernel values.
Tags 254 andd 255 are special markers for user-defined values.


%%%%%%%%%%%%%%% Section Immediate Values %%%%%%%%%%%
\section{Immediate Values}

\subsection{No value}

The no-value is represented by the tag $1$ and has no further information attached to it.

The representation is as follows:

\begin{tabular}{|c|c|c|}
\hline
\multirow{3}{*}{\code{novalue}} & 1 & (empty) \\
\cline{2-3}
& \code{tag} & \code{ctrl} \\
\cline{2-3}
& \multicolumn{2}{|c|}{1 word} \\
\hline
\end{tabular}


\subsection{Boolean values}

Here, a single machine word is enough to represent the whole value.

The value false has tag $2$ and information $0$.

\begin{tabular}{|c|c|c|}
\hline
\multirow{3}{*}{\code{false}} & 2 & 0 \\
\cline{2-3}
& \code{tag} & \code{ctrl} \\
\cline{2-3}
& \multicolumn{2}{|c|}{1 word} \\
\hline
\end{tabular}

The value true has tag $2$ and information $1$.

\begin{tabular}{|c|c|c|}
\hline
\multirow{3}{*}{\code{true}} & 2 & 1 \\
\cline{2-3}
& \code{tag} & \code{ctrl} \\
\cline{2-3}
& \multicolumn{2}{|c|}{1 word} \\
\hline
\end{tabular}


\subsection{Integer values}

The default integer type is the machine-word signed integer representation (i.e. int in C).

The representation is:

\begin{tabular}{|c|c|c|c|}
\hline
\multirow{3}{*}{\code{int(n)}} & 3 & (empty) & \code{n} \\
\cline{2-4}
 & \code{tag} & \code{ctrl} & \code{data} \\
\cline{2-4}
 & \multicolumn{2}{|c|}{1 word} & 1 word \\
\hline
\end{tabular}

\subsection{Float values}

The default float type is the double-precision IEEE FP representation (i.e. double in C).

\begin{tabular}{|c|c|c|c|}
\hline
\multirow{3}{*}{\code{float(x)}} & 3 & (empty) & \code{x} \\
\cline{2-4}
 & \code{tag} & \code{ctrl} & \code{data} \\
\cline{2-4}
 & \multicolumn{2}{|c|}{1 word} & 2 words \\
\hline
\end{tabular}



 part contains the number of
references to the string, and the info part contains the C-like representation of
 the string.   Note that all string 

%%%%%%%%%%%%%%% Section Tuple values %%%%%%%%%%%
\section{Tuple Values}

For tuple values, the tag is 64,  the control part contains the number of
sub-values, and the info part contains pointers to the successive sub-values.

\begin{tabular}{|c|c|c|c|}
\hline
\multirow{3}{*}{\code{tuple(v$_1$,$\ldots$,v$_n$)}} & 64 & \code{n} & \code{\&v$_1$,$\ldots$,\&v$_n$} \\
\cline{2-4}
 & \code{tag} & \code{ctrl} & \code{data} \\
\cline{2-4}
 & \multicolumn{2}{|c|}{1 word} & $n$ words \\
\hline
\end{tabular}

\textbf{Remark}: the tuple must be allocated on the heap, but it is an immutable structure that
 adopts a ``by-copy'' semantics.

%%%%%%%%%%%%%%% Section String values %%%%%%%%%%%
\section{String Values (managed)}

For string values, the tag part is 128, the control is empty and the data
 part provides a handle that allows to obtain the string representation.

\begin{tabular}{|c|c|c|c|}
\hline
\multirow{3}{*}{\code{string(h)}} & 128 & (empty) & \code{h} \\
\cline{2-4}
 & \code{tag} & \code{ctrl} & \code{data} \\
\cline{2-4}
 & \multicolumn{2}{|c|}{1 word} & 1 word \\
\hline
\end{tabular}

The string representation itself should be obtained by the handle \code{h}.
Note that the handle must be managed by the \code{knows} set because of
 alias-free reference counting for strings (cf. channel).

For the representation of the string, the encoding is UTF-8 and a reference
count is added. The string is \code{NULL}-terminated as in C.

\begin{tabular}{|c|c|c|}
\hline
\multirow{3}{*}{\code{string\_repr(str)}} & \code{refcount} & \code{str} \\
\cline{2-3}
 & 1 word & $\geq 1$ word(s) \\
\hline
\end{tabular}

\textbf{Remark}: an external library e.g. \code{utf8proc} can be considered
for supporting strings.

%%%%%%%%%%%%%%% Section Channel values %%%%%%%%%%%
\section{Channel Values (managed)}

The tag is $253$ and the control part gives the the
kind of channel  (standard, I/O, linear channel, etc.). The data part is a 
handle allowing to fetch the channel representation.

\begin{tabular}{|c|c|c|c|}
\hline
\multirow{3}{*}{\code{channel(k,h)}} & 128 & k & \code{h} \\
\cline{2-4}
 & \code{tag} & \code{ctrl} & \code{data} \\
\cline{2-4}
 & \multicolumn{2}{|c|}{1 word} & 1 word \\
\hline
\end{tabular}

The channel representation is described in the compilation specification.
The default channel kind is \code{PI\_CHANNEL} but other channel kinds can be
added to the runtime with different semantics.

%%%%%%%%%%%%%%% Section User-defined values %%%%%%%%%%%
\section{User-defined Values}

User-defined types are not yet specified, but the tags $254$ and $255$ must
be reserved for that purpose.

\subsection{immediate values}

The tag part is $254$, the control part contains the sub-type descriptor
 and the data part is user-defined.

\subsection{managed values}

The tag part is $255$, the control part contains the sub-type descriptor
 and the data part is a handle allowing to the fetch the representation part,
 which is user-specific.


\end{document}
